%\documentclass[twocolumn,journal]{IEEEtran}
\documentclass[onecolumn,journal]{IEEEtran}
\usepackage{amsfonts}
\usepackage{amsmath}
\usepackage{amsthm}
\usepackage{amssymb}
\usepackage{graphicx}
\usepackage[T1]{fontenc}
%\usepackage[english]{babel}
\usepackage{supertabular}
\usepackage{longtable}
\usepackage[usenames,dvipsnames]{color}
\usepackage{bbm}
%\usepackage{caption}
\usepackage{fancyhdr}
\usepackage{breqn}
\usepackage{fixltx2e}
\usepackage{capt-of}
%\usepackage{mdframed}
\setcounter{MaxMatrixCols}{10}
\usepackage{tikz}
\usetikzlibrary{matrix}
\usepackage{endnotes}
\usepackage{soul}
\usepackage{marginnote}
%\newtheorem{theorem}{Theorem}
\newtheorem{lemma}{Lemma}
%\newtheorem{remark}{Remark}
%\newtheorem{error}{\color{Red} Error}
\newtheorem{corollary}{Corollary}
\newtheorem{proposition}{Proposition}
\newtheorem{definition}{Definition}
\newcommand{\mathsym}[1]{}
\newcommand{\unicode}[1]{}
\newcommand{\dsum} {\displaystyle\sum}
\hyphenation{op-tical net-works semi-conduc-tor}
\usepackage{pdfpages}
\usepackage{enumitem}
\usepackage{multicol}

\headsep = 5pt
\textheight = 730pt
%\headsep = 8pt %25pt
%\textheight = 720pt %674pt
%\usepackage{geometry}

\bibliographystyle{unsrt}

\usepackage{float}

 \usepackage{xcolor}
 
\usepackage[framemethod=TikZ]{mdframed}
%%%%%%%FRAME%%%%%%%%%%%
\usepackage[framemethod=TikZ]{mdframed}
\usepackage{framed}
    % \BeforeBeginEnvironment{mdframed}{\begin{minipage}{\linewidth}}
     %\AfterEndEnvironment{mdframed}{\end{minipage}\par}


%	%\mdfsetup{%
%	%skipabove=20pt,
%	nobreak=true,
%	   middlelinecolor=black,
%	   middlelinewidth=1pt,
%	   backgroundcolor=purple!10,
%	   roundcorner=1pt}

\mdfsetup{%
	outerlinewidth=1,skipabove=20pt,backgroundcolor=yellow!50, outerlinecolor=black,innertopmargin=0pt,splittopskip=\topskip,skipbelow=\baselineskip, skipabove=\baselineskip,ntheorem,roundcorner=5pt}

\mdtheorem[nobreak=true,outerlinewidth=1,%leftmargin=40,rightmargin=40,
backgroundcolor=yellow!50, outerlinecolor=black,innertopmargin=0pt,splittopskip=\topskip,skipbelow=\baselineskip, skipabove=\baselineskip,ntheorem,roundcorner=5pt,font=\itshape]{result}{Result}


\mdtheorem[nobreak=true,outerlinewidth=1,%leftmargin=40,rightmargin=40,
backgroundcolor=yellow!50, outerlinecolor=black,innertopmargin=0pt,splittopskip=\topskip,skipbelow=\baselineskip, skipabove=\baselineskip,ntheorem,roundcorner=5pt,font=\itshape]{theorem}{Theorem}

\mdtheorem[nobreak=true,outerlinewidth=1,%leftmargin=40,rightmargin=40,
backgroundcolor=gray!10, outerlinecolor=black,innertopmargin=0pt,splittopskip=\topskip,skipbelow=\baselineskip, skipabove=\baselineskip,ntheorem,roundcorner=5pt,font=\itshape]{remark}{Remark}

\mdtheorem[nobreak=true,outerlinewidth=1,%leftmargin=40,rightmargin=40,
backgroundcolor=pink!30, outerlinecolor=black,innertopmargin=0pt,splittopskip=\topskip,skipbelow=\baselineskip, skipabove=\baselineskip,ntheorem,roundcorner=5pt,font=\itshape]{quaestio}{Quaestio}

\mdtheorem[nobreak=true,outerlinewidth=1,%leftmargin=40,rightmargin=40,
backgroundcolor=yellow!50, outerlinecolor=black,innertopmargin=5pt,splittopskip=\topskip,skipbelow=\baselineskip, skipabove=\baselineskip,ntheorem,roundcorner=5pt,font=\itshape]{background}{Background}

%TRYING TO INCLUDE Ppls IN TOC
\usepackage{hyperref}


\begin{document}
\title{\color{Brown}  Massive Testing Can Stop the Coronavirus Outbreak \\
\vspace{-0.35ex}}
\author{Chen Shen and Yaneer Bar-Yam \\ New England Complex Systems Institute \\
 \today 
  \vspace{-8ex} \\ 
%\bigskip
\textbf{}
 }
    
\maketitle

%\vspace{-1ex}
%\flushbottom % Makes all text pages the same height

%\maketitle % Print the title and abstract box

%\tableofcontents % Print the contents section

\thispagestyle{empty} % Removes page numbering from the first page

%----------------------------------------------------------------------------------------
%	ARTICLE CONTENTS
%----------------------------------------------------------------------------------------

%\section*{Introduction} % The \section*{} command stops section numbering

%\addcontentsline{toc}{section}{\hspace*{-\tocsep}Introduction} % Adds this section to the table of contents with negative horizontal space equal to the indent for the numbered sections

%\tableofcontents 
%\section{ Introduction}

%\section*{Overview}
%\begin{multicols}

\begin{multicols}{2}

In order to stop an outbreak transmission has to be stopped. A key strategy is to identify individuals who have the disease and isolate them so that others are not infected. %In order to isolate those individuals we need to identify them. 
If the test is not perfect we might isolate additional people who are not infected---these are false positives. This leads to additional social cost but will still stop the outbreak. On the other hand, we can also allow a number of false negatives. As long as the ratio of false negatives is small enough, there will be fewer new cases over time and the outbreak will die out exponentially. The greater the number of individuals each sick individual infects (infection rate or reproduction rate), the smaller the allowable false negative rate.

The more specific we can be about identifying potential cases, the better, because fewer people have to be isolated. On the other hand, the fewer the cases we miss even if we isolate more people, the more rapid the outbreak disappears, and the fewer people get sick and die. 

How do the various ways we stop outbreaks relate to this general framework? Here are a few examples:
\begin{itemize}
\item \textbf{Self-reporting and diagnosis:} In this way of testing, an individual first has to identify they have symptoms that require medical care, and then report to a physician who performs a diagnosis, and if the diagnosis determines that they have this particular disease (with some false positives and negatives) the individual is isolated. Those who are sick and don't self-report are false-negatives. Those who are incorrectly diagnosed with the disease are false-positives. Typically, the most important difficulty is false-negatives because of self-reporting: people who are sick, but do not recognize it and don't self-report, perhaps because symptoms are generic/nonspecific, or immediately life threatening. Alternatively, individuals may suspect they have the disease but for personal, economic, social or professional reasons don't choose to be diagnosed, or are not provided the opportunity to be diagnosed, and isolated. (There are other issues including whether the process of going to get tested, being tested, and being isolated results in new cases, e.g. infecting those during transportation or in the medical offices, and how successful is the isolation). 

\item \textbf{Contact tracing:} In this way of testing, individuals who have been in contact with a diagnosed individual (according to the self-reporting and diagnosis method) are identified and are contacted to either be on the lookout for symptoms or directly to be isolated. Even if they are not infected, their isolation (with many people who are not actually infected, i.e. false positives) is used to stop the outbreak.  

\item \textbf{Lock down---geographic community identification:} In this way of testing, all members of a geographic community in the area of infected individuals are considered to be potentially infected and isolated. This includes many false positives, and can stop the outbreak. 

\item \textbf{Neighborhood generic symptomatic testing:} In this way of testing, all members of a geographic community in the area of infected individuals are further tested for symptoms such as fever that may be associated with the disease, but may also be associated with other conditions, and are considered to be potentially infected and isolated. The advantage of this approach over lockdowns, is that fewer individuals are isolated, reducing social cost. The advantage over more specific diagnosis is that many more individuals who are infected are isolated. This approach was used effectively to stop the Ebola outbreak in Liberia and Sierra Leone \cite{ebola}.

\item \textbf{Massive specific testing:} in this way of testing, DNA or other specific tests, are applied widely to the population, perhaps focused on a specific geographical area, in order to identify potential cases to be isolated. If the test is specific enough and can be applied widely enough, this approach can stop the outbreak. 

\item \textbf{Targeted random sampling:} in this way of testing, diagnostic or DNA tests, are applied to individuals of populations that are highly connected, for example, in confined communities such as prisons, dormitories, hostels, nursing homes, rehabilitation facilities, psychiatric wards, medical facilities, or retirement communities. In those locations when one individual is infected many individuals are likely to be infected even if they are not yet showing symptoms or would have positive test results. In that case the entire community can be isolated (as individuals and not as a group) to prevent additional transmission. 

\end{itemize}

More generally, we see that any way of determining the individuals to be isolated, using various tests including symptoms, geographic location, or molecular tests, that have the ability to identify those who are infected, even if there are false positives, and with few enough false negatives, can be used to stop a contagion.

In addition to performing the testing, a key question is how early can we identify if someone is a member of the group that should be isolated and how does that affect the number of people they infect. If they are identified before they become contagious, or there is only a small window of time they are contagious, that can be sufficient in preventing the infections to be effective at stopping the outbreak. The fraction of time they are contagious acts in a similar way to false negatives, it contributes to the number of individuals that are infected and reduces the effectiveness of the test for stopping the outbreak. This makes early and rapid application of the test, no matter what form it takes, symptomatic, geographic, or molecular, an essential part of whether the test is effective in stopping the outbreak, and how many people become sick and die. 

In the case of COVID-19, the Coronavirus outbreak starting in Wuhan, there is a specific DNA test using a nasal or throat swab, that can provide rapid enough identification of cases, to reduce dramatically the contagion rate if those who test positive are isolated. The tests take several days to provide results. More rapid ones are in development. 

Early in the outbreak, there was a severe limit to the number of tests that could be done, and the need for community based isolation was an imperative, and was used in China to remarkable success augmenting, particularly in Wuhan, a massive more traditional contact tracing effort (670,000 people) \cite{China}. Subsequently in South Korea, while a lock down was implemented  \cite{Daegu}, so was much larger scale testing, including convenient drive by locations \cite{SouthKorea}. Very recently there is indication that the outbreak in South Korea is under control \cite{successSK}.

At this time in many places in the world, including the US, there are insufficient tests to achieve widespread testing. This limits our ability to use this approach. Still, it is possible for the test to be produced rapidly and cheaply and then applied massively to identify cases as an alternative to community lock-down. Once test are available, massive specific testing can achieve the desired outcome of stopping the outbreak.

What should be done? Slowing or stopping the Coronavirus outbreak can take a multi-pronged approach. Individuals, families and communities should take precautions to avoid contacts and limit their and others likelihood of becoming infected. At the same time medical authorities should ramp up testing with minimal qualifications and ready access across geographical locations. Where it is possible, corporations or NGOs may also provide test services in convenient locations, perhaps even door to door, as was done toward the tail end of the outbreak in China \cite{cdoor}, for rapid identification of cases. 

\end{multicols}

\vspace{2ex}

\begin{thebibliography}{20}

\bibitem{ebola} How community response stopped ebola \url{https://necsi.edu/how-community-response-stopped-ebola}
\bibitem{China} Report of the WHO-China Joint Mission on Coronavirus Disease 2019 (COVID-19) \url{https://www.who.int/docs/default-source/coronaviruse/who-china-joint-mission-on-covid-19-final-report.pdf}
\bibitem{Daegu} Daegu in Lockdown as Coronavirus Infections Soar \url{http://english.chosun.com/site/data/html_dir/2020/02/24/2020022401353.html}
\bibitem{SouthKorea} South Korea pioneers coronavirus drive-through testing station \url{https://www.cnn.com/2020/03/02/asia/coronavirus-drive-through-south-korea-hnk-intl/index.html}
\bibitem{successSK} BREAKING: Coronavirus Update. Significant decline in daily new cases in South Korea--positive sign of gaining control. \url{https://twitter.com/yaneerbaryam/status/1235734017699430401?s=20}
\bibitem{cdoor} China Goes Door to Door in Wuhan, Seeking Infections \url{https://www.courthousenews.com/china-goes-door-to-door-in-wuhan-seeking-infections/}

\end{thebibliography}


\end{document}